\documentclass[12pt,oneside,a4paper]{article}
\usepackage[table]{xcolor}
\usepackage{graphicx}
\usepackage{amsmath}
\usepackage{fancyhdr}


\pagestyle{fancy}
\cfoot{\thepage}
\renewcommand*\contentsname{Sommaire}

\begin{document}
\title{Report}
\author{Fanny Kalinowski, Julien Molinier, Robin Lambert, Maxime Leras}
\date{25 Mars 2020}
\maketitle
\newpage    
\tableofcontents

\newpage
\pagenumbering{arabic}
\section{Introduction}
\paragraph{}
    Tabu Search is an algorithm created in 1986. It's an heuristic employing
    local search methods. The goal is to take a potential solution of a problem
    and check its immediate neighbors in the hope of finding an improved solution.
    \footnote{https://en.wikipedia.org/wiki/Tabu\_search}
\paragraph{}
    The main objective of this report is to analyse an implementation of
    the Tabu Search algorithm and see the effect when changing the paramaters values.




\section{Tests without Tabu duration on 10 cities}
\subsection{Run of the algorithm}
\paragraph{}
    First of all we ran the algorithm 10 times for 5, 10 and 100 iterations. The algorithm stop when he found
    the global minimum witch is supposed to be 3473km. The results are following in the table below :    

\begin{table}[h]
    \centering
    \small
    \begin{tabular}{llll}
      \hline
      \multicolumn{1}{|l|}{\textbf{Run / Nb\_iterations}}& \multicolumn{1}{l|}{\textbf{5}} & \multicolumn{1}{l|}{\textbf{10}} & \multicolumn{1}{l|}{\textbf{100}}\\ \hline
      \multicolumn{1}{|l|}{1} & \multicolumn{1}{l|}{-}  & \multicolumn{1}{l|}{5}  & \multicolumn{1}{l|}{4}  \\ \hline
      \multicolumn{1}{|l|}{2} & \multicolumn{1}{l|}{-}  & \multicolumn{1}{l|}{7}  & \multicolumn{1}{l|}{4}  \\ \hline         
      \multicolumn{1}{|l|}{3} & \multicolumn{1}{l|}{-}  & \multicolumn{1}{l|}{5}  & \multicolumn{1}{l|}{4}  \\ \hline
      \multicolumn{1}{|l|}{4} & \multicolumn{1}{l|}{-}  & \multicolumn{1}{l|}{4}  & \multicolumn{1}{l|}{6}  \\ \hline
      \multicolumn{1}{|l|}{5} & \multicolumn{1}{l|}{-}  & \multicolumn{1}{l|}{6}  & \multicolumn{1}{l|}{4}  \\ \hline
      \multicolumn{1}{|l|}{6} & \multicolumn{1}{l|}{3}  & \multicolumn{1}{l|}{5}  & \multicolumn{1}{l|}{7}  \\ \hline
      \multicolumn{1}{|l|}{7} & \multicolumn{1}{l|}{-}  & \multicolumn{1}{l|}{7}  & \multicolumn{1}{l|}{4}  \\ \hline
      \multicolumn{1}{|l|}{8} & \multicolumn{1}{l|}{-}  & \multicolumn{1}{l|}{7}  & \multicolumn{1}{l|}{4}  \\ \hline
      \multicolumn{1}{|l|}{9} & \multicolumn{1}{l|}{-}  & \multicolumn{1}{l|}{5}  & \multicolumn{1}{l|}{7}  \\ \hline
      \multicolumn{1}{|l|}{10} & \multicolumn{1}{l|}{-}  & \multicolumn{1}{l|}{3}  & \multicolumn{1}{l|}{5}  \\ \hline
    \end{tabular}
    \caption{Number of iteration needed without tabu duration on 10 cities}
    \label{Performances scénario 0}
  \end{table}

\paragraph{}
    From this table, the computation of the average gives 4.9 on 10 iterations et 4.9 on 100 iterations.
    It means the algorithm need around 5 iterations to find the global minimum and it explains why the algorithm
    ended only 1 time when the Nb\_iterations is equal to 5.

\subsection{Convergence}
\paragraph{}
    The values computed in section 2.1 shows that the algorithm needs 5 iterations on average to converge to the
    global minimum. After the algorithm computes the neighbors of this solution but as the previous solution is the
    global minimum the neighbors won't be better. So the global minimum remains the same and the algorithm repeats this step
    again and again until the maximum number of iterations sets before launching. That's why when the maximum is 5 the
    algorithm can't always find the global minimum.

\subsection{Number of solutions and neighbors with 2-opt}
\paragraph{}
    For 10 cities the number of solutions is 9! and for 50 cities it is 49!.
\paragraph{}
    The number of neighbors with 2-opt follows the formula :
    \[\binom{n}{1}\binom{n-3}{1}/2\]
\paragraph{}
    For 10 cities it give 35 and for 50 cities 1175 neighbors.

\subsection{Algorithm performance}
\paragraph{}
    The number of solutions visited by the algorithm before the convergence is given by the formula
    \textit{Nb\_iterations * Nb\_neighbors}. So with 10 cities if the algorithm find the solution at the 
    iteration 4 it means that 40 solutions were visited. Compared to the 9! possible ones the algorithm tries only 1
    out of 10000 solutions witch is pretty good.


\section{Tests without Tabu duration on 50 cities}
\paragraph{}
\subsection{Run of the algorithm}
\begin{table}[h]
    \centering
    \small
    \begin{tabular}{llll}
      \hline
      \multicolumn{1}{|l|}{\textbf{Run / Nb\_iterations}}& \multicolumn{1}{l|}{\textbf{5}} & \multicolumn{1}{l|}{\textbf{10}} & \multicolumn{1}{l|}{\textbf{100}}\\ \hline
      \multicolumn{1}{|l|}{1} & \multicolumn{1}{l|}{-}  & \multicolumn{1}{l|}{44}  & \multicolumn{1}{l|}{50}  \\ \hline
      \multicolumn{1}{|l|}{2} & \multicolumn{1}{l|}{-}  & \multicolumn{1}{l|}{44}  & \multicolumn{1}{l|}{47}  \\ \hline         
      \multicolumn{1}{|l|}{3} & \multicolumn{1}{l|}{-}  & \multicolumn{1}{l|}{48}  & \multicolumn{1}{l|}{43}  \\ \hline
      \multicolumn{1}{|l|}{4} & \multicolumn{1}{l|}{-}  & \multicolumn{1}{l|}{48}  & \multicolumn{1}{l|}{46}  \\ \hline
      \multicolumn{1}{|l|}{5} & \multicolumn{1}{l|}{-}  & \multicolumn{1}{l|}{40}  & \multicolumn{1}{l|}{44}  \\ \hline
      \multicolumn{1}{|l|}{6} & \multicolumn{1}{l|}{-}  & \multicolumn{1}{l|}{42}  & \multicolumn{1}{l|}{44}  \\ \hline
      \multicolumn{1}{|l|}{7} & \multicolumn{1}{l|}{-}  & \multicolumn{1}{l|}{44}  & \multicolumn{1}{l|}{47}  \\ \hline
      \multicolumn{1}{|l|}{8} & \multicolumn{1}{l|}{-}  & \multicolumn{1}{l|}{43}  & \multicolumn{1}{l|}{42}  \\ \hline
      \multicolumn{1}{|l|}{9} & \multicolumn{1}{l|}{-}  & \multicolumn{1}{l|}{42}  & \multicolumn{1}{l|}{43}  \\ \hline
      \multicolumn{1}{|l|}{10} & \multicolumn{1}{l|}{-}  & \multicolumn{1}{l|}{44}  & \multicolumn{1}{l|}{47}  \\ \hline
    \end{tabular}
    \caption{Number of iteration needed without tabu duration on 50 cities}
    \label{Performances scénario 0}
  \end{table}

Définition des défis et les solutions apportées ou envisagées

\section{Tests with Tabu duration}

\section{Variation in Tabu duration}

\section{Conclusion}
\paragraph{}

\end{document}