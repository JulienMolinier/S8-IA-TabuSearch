\documentclass[12pt,oneside,a4paper]{article}
\usepackage[table]{xcolor}
\usepackage{graphicx}
\usepackage{amsmath}
\usepackage{fancyhdr}


\pagestyle{fancy}
\cfoot{\thepage}
\renewcommand*\contentsname{Sommaire}

\begin{document}
\title{Report}
\author{Fanny Kalinowski, Julien Molinier, Robin Lambert, Maxime Leras}
\date{25 Mars 2020}
\maketitle
\newpage    
\tableofcontents

\newpage
\pagenumbering{arabic}
\section{Introduction}
\paragraph{}
    Tabu Search is an algorithm created in 1986. It's an heuristic employing
    local search methods. The goal is to take a potential solution of a problem
    and check its immediate neighbors in the hope of finding an improved solution.
    \footnote{https://en.wikipedia.org/wiki/Tabu\_search}
\paragraph{}
    The main objective of this report is to analyse an implementation of
    the Tabu Search algorithm and see the effect when changing the paramaters values.




\section{Tests without Tabu duration on 10 cities}
\subsection{Run of the algorithm}
\paragraph{}
    First of all we ran the algorithm 10 times for 5, 10 and 100 iterations. The algorithm stop when he found
    the global minimum witch is supposed to be 3473km. The results are following in the table below :    

\begin{table}[h]
    \centering
    \small
    \begin{tabular}{llll}
      \hline
      \multicolumn{1}{|l|}{\textbf{Run / Nb\_iterations}}& \multicolumn{1}{l|}{\textbf{5}} & \multicolumn{1}{l|}{\textbf{10}} & \multicolumn{1}{l|}{\textbf{100}}\\ \hline
      \multicolumn{1}{|l|}{1} & \multicolumn{1}{l|}{-}  & \multicolumn{1}{l|}{5}  & \multicolumn{1}{l|}{4}  \\ \hline
      \multicolumn{1}{|l|}{2} & \multicolumn{1}{l|}{-}  & \multicolumn{1}{l|}{7}  & \multicolumn{1}{l|}{4}  \\ \hline         
      \multicolumn{1}{|l|}{3} & \multicolumn{1}{l|}{-}  & \multicolumn{1}{l|}{5}  & \multicolumn{1}{l|}{4}  \\ \hline
      \multicolumn{1}{|l|}{4} & \multicolumn{1}{l|}{-}  & \multicolumn{1}{l|}{4}  & \multicolumn{1}{l|}{6}  \\ \hline
      \multicolumn{1}{|l|}{5} & \multicolumn{1}{l|}{-}  & \multicolumn{1}{l|}{6}  & \multicolumn{1}{l|}{4}  \\ \hline
      \multicolumn{1}{|l|}{6} & \multicolumn{1}{l|}{3}  & \multicolumn{1}{l|}{5}  & \multicolumn{1}{l|}{7}  \\ \hline
      \multicolumn{1}{|l|}{7} & \multicolumn{1}{l|}{-}  & \multicolumn{1}{l|}{7}  & \multicolumn{1}{l|}{4}  \\ \hline
      \multicolumn{1}{|l|}{8} & \multicolumn{1}{l|}{-}  & \multicolumn{1}{l|}{7}  & \multicolumn{1}{l|}{4}  \\ \hline
      \multicolumn{1}{|l|}{9} & \multicolumn{1}{l|}{-}  & \multicolumn{1}{l|}{5}  & \multicolumn{1}{l|}{7}  \\ \hline
      \multicolumn{1}{|l|}{10} & \multicolumn{1}{l|}{-}  & \multicolumn{1}{l|}{3}  & \multicolumn{1}{l|}{5}  \\ \hline
    \end{tabular}
    \caption{Number of iteration needed without tabu duration on 10 cities}
    \label{Performances scénario 0}
  \end{table}

\paragraph{}
    From this table, the computation of the average gives 4.9 on 10 iterations et 4.9 on 100 iterations.
    It means the algorithm need around 5 iterations to find the global minimum and it explains why the algorithm
    ended only 1 time when the Nb\_iterations is equal to 5.

\subsection{Convergence}
\paragraph{}
    The values computed in section 2.1 shows that the algorithm needs 5 iterations on average to converge to the
    global minimum. After the algorithm computes the neighbors of this solution but as the previous solution is the
    global minimum the neighbors won't be better. So the global minimum remains the same and the algorithm repeats this step
    again and again until the maximum number of iterations sets before launching. That's why when the maximum is 5 the
    algorithm can't always find the global minimum.

\subsection{Number of solutions and neighbors with 2-opt}
\paragraph{}
    For 10 cities the number of solutions is 9! and for 50 cities it is 49!.
\paragraph{}
    The number of neighbors with 2-opt follows the formula :
    \[\binom{n}{1}\binom{n-3}{1}/2\]
\paragraph{}
    For 10 cities it give 35 and for 50 cities 1175 neighbors.

\subsection{Algorithm performance}
\paragraph{}
    The number of solutions visited by the algorithm before the convergence is given by the formula
    \textit{Nb\_iterations * Nb\_neighbors}. So with 10 cities if the algorithm find the solution at the 
    iteration 4 it means that 40 solutions were visited. Compared to the 9! possible ones the algorithm tries only 1
    out of 10000 solutions witch is pretty good.
\newpage

\section{Tests without Tabu duration on 50 cities}
\subsection{Run of the algorithm}
\begin{table}[h]
    \centering
    \small
    \begin{tabular}{llll}
      \hline
      \multicolumn{1}{|l|}{\textbf{Run / Nb\_iterations}}& \multicolumn{1}{l|}{\textbf{10}} & \multicolumn{1}{l|}{\textbf{100}} & \multicolumn{1}{l|}{\textbf{1000}}\\ \hline
      \multicolumn{1}{|l|}{1} & \multicolumn{1}{l|}{-}  & \multicolumn{1}{l|}{44}  & \multicolumn{1}{l|}{50}  \\ \hline
      \multicolumn{1}{|l|}{2} & \multicolumn{1}{l|}{-}  & \multicolumn{1}{l|}{44}  & \multicolumn{1}{l|}{47}  \\ \hline         
      \multicolumn{1}{|l|}{3} & \multicolumn{1}{l|}{-}  & \multicolumn{1}{l|}{48}  & \multicolumn{1}{l|}{43}  \\ \hline
      \multicolumn{1}{|l|}{4} & \multicolumn{1}{l|}{-}  & \multicolumn{1}{l|}{48}  & \multicolumn{1}{l|}{46}  \\ \hline
      \multicolumn{1}{|l|}{5} & \multicolumn{1}{l|}{-}  & \multicolumn{1}{l|}{40}  & \multicolumn{1}{l|}{44}  \\ \hline
      \multicolumn{1}{|l|}{6} & \multicolumn{1}{l|}{-}  & \multicolumn{1}{l|}{42}  & \multicolumn{1}{l|}{44}  \\ \hline
      \multicolumn{1}{|l|}{7} & \multicolumn{1}{l|}{-}  & \multicolumn{1}{l|}{44}  & \multicolumn{1}{l|}{47}  \\ \hline
      \multicolumn{1}{|l|}{8} & \multicolumn{1}{l|}{-}  & \multicolumn{1}{l|}{43}  & \multicolumn{1}{l|}{42}  \\ \hline
      \multicolumn{1}{|l|}{9} & \multicolumn{1}{l|}{-}  & \multicolumn{1}{l|}{42}  & \multicolumn{1}{l|}{43}  \\ \hline
      \multicolumn{1}{|l|}{10} & \multicolumn{1}{l|}{-}  & \multicolumn{1}{l|}{44}  & \multicolumn{1}{l|}{47}  \\ \hline
    \end{tabular}
    \caption{Number of iteration needed without tabu duration on 50 cities}
    \label{Performances scénario 0}
  \end{table}
\paragraph{}
    For 10 iterations, the algorithm never found the global minimum. On average, the global minimum is found at the 43.9th for 100 iterations. For 1000 iterations, the global minimum is found at the 45,3th iteration.

\subsection{Convergence}
\paragraph{}
The algorithm needs around 43,9 iterations to converge for Nb\_iterations = 100 iterations, and 45,3 iterations to converge for Nb\_iterations = 1000 iterations. 
\paragraph{}
    Once the algorithm have found the global minimum, it doesn't explore other paths In the following iterations, the value of the kilometers is still blocked on the global minimum. According to the algorithm, the global minima is the best solution until yet, so there is no need to explore the neighbour around the top.  
This is due to the Duration\_tabou which is equal to 0. This means that the number of solution that have been explored successively are not kept in memory. Thus, the same points are explored several times. Consequently,  the local minima is not going to change.

\subsection{Solutions visited before the convergence}
\paragraph{}
    The algorithm starts to converge once it has found a local minima.
So, in order to know the number of solutions visited before its convergence we need to count all the solutions that have been explored before the finding of the local minima. 
In the first run, with Nb\_iterations = 100, 43 solutions have been explored before the finding of the local minima. So, for each of them, the best solution is displayed. It’s the result of the exploration of all the 1175 neighbors.
\begin{table}[h]
    \centering
    \small
    \begin{tabular}{llll}
      \hline
      \multicolumn{1}{|l|}{\textbf{Run / Nb\_iterations}}& \multicolumn{1}{l|}{\textbf{10}} & \multicolumn{1}{l|}{\textbf{100}} & \multicolumn{1}{l|}{\textbf{1000}}\\ \hline
      \multicolumn{1}{|l|}{1} & \multicolumn{1}{l|}{-}  & \multicolumn{1}{l|}{50525}  & \multicolumn{1}{l|}{57575}  \\ \hline
      \multicolumn{1}{|l|}{2} & \multicolumn{1}{l|}{-}  & \multicolumn{1}{l|}{50525}  & \multicolumn{1}{l|}{54060}  \\ \hline         
      \multicolumn{1}{|l|}{3} & \multicolumn{1}{l|}{-}  & \multicolumn{1}{l|}{55225}  & \multicolumn{1}{l|}{49350}  \\ \hline
      \multicolumn{1}{|l|}{4} & \multicolumn{1}{l|}{-}  & \multicolumn{1}{l|}{55225}  & \multicolumn{1}{l|}{52875}  \\ \hline
      \multicolumn{1}{|l|}{5} & \multicolumn{1}{l|}{-}  & \multicolumn{1}{l|}{45825}  & \multicolumn{1}{l|}{50525}  \\ \hline
      \multicolumn{1}{|l|}{6} & \multicolumn{1}{l|}{-}  & \multicolumn{1}{l|}{48175}  & \multicolumn{1}{l|}{50525}  \\ \hline
      \multicolumn{1}{|l|}{7} & \multicolumn{1}{l|}{-}  & \multicolumn{1}{l|}{50525}  & \multicolumn{1}{l|}{54060}  \\ \hline
      \multicolumn{1}{|l|}{8} & \multicolumn{1}{l|}{-}  & \multicolumn{1}{l|}{49350}  & \multicolumn{1}{l|}{48175}  \\ \hline
      \multicolumn{1}{|l|}{9} & \multicolumn{1}{l|}{-}  & \multicolumn{1}{l|}{48175}  & \multicolumn{1}{l|}{49350}  \\ \hline
      \multicolumn{1}{|l|}{10} & \multicolumn{1}{l|}{-}  & \multicolumn{1}{l|}{48175}  & \multicolumn{1}{l|}{54060}  \\ \hline
    \end{tabular}
    \caption{Number of solutions visited before the convergence without tabu duration on 50 cities}
    \label{Performances scénario 0}
  \end{table}
\paragraph{}
  Average à calculer
  On average, 42.7 solutions have been explored before finding the local minima for Nb\_iterations = 100, against 44.3 explored solutions for Nb\_iterations = 1000. The two runs show that the local minima is found between the 42.7th and the 44.3th iteration. 

This is logical, since the Duration\_tabou is equal to 0 in the two runs. 
Increasing the number of iterations is interesting if the value of the Duration\_tabou changes. In fact, it will allow the algorithm not to go through the same path and to have more time (thanks to the iterations) to explore.
If the value of the Duration\_tabou doesn’t change, once the algorithm have found the local minima, it will start to converge to it, no matters the number of iterations. 
So here, the number of iterations does not influence the local minima.

On 100 iterations, the local minima is found at the 43th iteration. In the 57 following iterations, the algorithm converge to this local minima. That’s to say that during 57 iterations there is not relevant information, as the local minima is not going to change. 
On 1000 iterations, the local minima is found at the 44th iteration. So, there is a convergence during the following 956 iterations, meaning that during 956 iterations there is not relevant information. 

Since the time loss is less important on 100 iterations than on 1000 iterations, the one can say that running the algorithm on 100 iterations allows to gain performance.

\section{Tests with Tabu duration}

\section{Variation in Tabu duration}

\section{Conclusion}
\paragraph{}

\end{document}